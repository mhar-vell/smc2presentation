%*----------- SLIDE -------------------------------------------------------------
\begin{frame}[t]{Contexto} 
    \transdissolve[duration=0.5]
    A pesquisa desta tese visa desenvolver um modelo para a compensação das perturbações sofridas por manipuladores utilizados em veículos submarinos remotamente controlados, buscando dessa forma uma maior eficiência no planejamento e realização de trajetórias específicas de atividades submarinas.\\
    \vspace*{0.2cm}
    Pontos cruciais para o setor de atividades submarinas utilizando veículos submarinos:
    \newline
        \begin{columns}[c]
            \column{.05\textwidth}
            \column{.35\textwidth}
                \begin{enumerate}
                    \item tempo de realização da atividade;
                    \item complexidade da operação;
                    \item eficiência da tarefa.
                \end{enumerate}
            \column{.6\textwidth}
            \includemedia[
                width=0.7\linewidth,
                totalheight=0.39375\linewidth,
                activate=pageopen,
                passcontext, 
                addresource=./Media/movies/rov-fixing.mp4,
                flashvars={
                source=./Media/movies/rov-fixing.mp4
                &autoPlay=true
                &Loop=false}
                ]{\fbox{\includegraphics{rov}}}{VPlayer.swf}
        \end{columns}
%*----------- notes
    \note[item]{Notes can help you to remember important information. Turn on the notes option.}
\end{frame}
%-
%*----------- SLIDE -------------------------------------------------------------
\begin{frame}[c]{Questões de Pesquisa}
    \transboxout[duration=0.5]
    \begin{columns}
        \column{.05\textwidth}
        \column{.40\textwidth}
            \includegraphics[width=1.1\textwidth]{blueprint-arm}
        \column{.55\textwidth}
            \begin{enumerate}
                \item De que forma as perturbações podem ser compensadas num manipulador submarino?
                \item Qual o modelo para uma melhor eficiência de trajetórias?
                \item Quais variáveis são preponderantes para um controle de trajetórias?
                \item Como estas variáveis podem interferir num novo modelo?
            \end{enumerate}
    \end{columns}
%*----------- notes
    \note[item]{Notes can help you to remember important information. Turn on the notes option.}
\end{frame}
%-
%*----------- SLIDE -------------------------------------------------------------
\begin{frame}[c]{Objetivo geral}
    %\transboxin[duration=1,direction=30]
    %\setbeamercolor{background canvas}{bg=yellow}
    \Wider{%
    \begin{shaded}
    \begin{center}
        \resizebox{!}{0.5cm}{%
            Propor um modelo dinâmico para
        }%
        \\
        \vspace*{0.5cm}
        \resizebox{!}{0.7cm}{%
            planejamento de trajetórias.
        }%
    \end{center}
    \end{shaded}
    }%

%*----------- notes
    \note[item]{Notes can help you to remember important information. Turn on the notes option.}
\end{frame}
%-
\begin{frame}[c]{Objetivos específicos}
    %\transboxin[duration=1,direction=30]
    \centering
    \begin{enumerate}
        \item Realizar \textbf{comparação} entre modelos existentes de planejamento de trajetórias.
        \item Implementar \textbf{odometria visual} num manipulador.
        \item \textbf{Integrar} a odometrial visual com o modelo de planejamento de trajetórias.
        \item \textbf{Simular o modelo} no sistema proposto do manipulador.
    \end{enumerate}
%*----------- notes
    \note[item]{Notes can help you to remember important information. Turn on the notes option.}
\end{frame}
%-
\begin{frame}[c]{Categorias Teóricas}
    %\transboxin[duration=1,direction=30]
    \centering
    \begin{columns}
        \column{.02\textwidth}
        \column{.48\textwidth}
        \begin{itemize}
            \item Planejamento dinâmico de trajetórias.
            \item Odometria visual.
            \item Manipuladores subaquáticos.
            \item Manipuladores autônomos.
            \item Manipuladores subaquáticos autônomos.
            \item Operação submarina autônoma.
        \end{itemize}
        \column{.25\textwidth}
            \centering
            \includegraphics[width=1\textwidth]{kraft}\\
            \includegraphics[width=.6\textwidth]{apriltag}
        \column{.25\textwidth}
            \centering
            \includegraphics[width=1\textwidth]{visualodom}\\
            \includegraphics[width=1\textwidth]{opsub}

        % \begin{figure}[ht] \label{ fig7} 
        %     \begin{minipage}[b]{0.5\linewidth}
        %       \includegraphics[width=.5\linewidth]{blueprint-arm} 
        %       \caption{Initial condition} 
        %     \end{minipage} 
        %     \begin{minipage}[b]{0.5\linewidth}
        %       \includegraphics[width=.5\linewidth]{blueprint-arm} 
        %       \caption{Rupture} 
        %     \end{minipage} 
        %     \begin{minipage}[b]{0.5\linewidth}
        %       \includegraphics[width=.5\linewidth]{blueprint-arm} 
        %       \caption{DFT, Initial condition} 
        %     \end{minipage}
        %     \hfill
        %     \begin{minipage}[b]{0.5\linewidth}
        %       \includegraphics[width=.5\linewidth]{blueprint-arm} 
        %       \caption{DFT, rupture} 
        %     \end{minipage} 
        %   \end{figure}
    \end{columns}
    % \begin{itemize}
    %     \item Planejamento de trajetórias.
    %     \item Odometria visual.
    %     \item Manipuladores subaquáticos.
    %     \item Manipuladores autônomos.
    %     \item Manipuladores subaquáticos autônomos.
    %     \item Operação submarina autônoma.
    % \end{itemize}
    % \begin{block}
    %     {MANIPULADOR}{manipulador}
    % \end{block}

%*----------- notes
    \note[item]{Notes can help you to remember important information. Turn on the notes option.}
\end{frame}
%-
\begin{frame}[c]{Metodologia}
    %\transboxin[duration=1,direction=30]
    \centering
    \includegraphics[width=0.7\textwidth]{metodologia11}
%*----------- notes
    \note[item]{Notes can help you to remember important information. Turn on the notes option.}
\end{frame}
%-
\begin{frame}[c]{Cronograma}
    %\transboxin[duration=1,direction=30]
    \centering
    % \begin{columns}
    %     \column{.05\textwidth}
    %     \column{.95\textwidth}
        % Please add the following required packages to your document preamble:
% \usepackage[table,xcdraw]{xcolor}
% If you use beamer only pass "xcolor=table" option, i.e. \documentclass[xcolor=table]{beamer}
\begin{table}[]
    \resizebox{\linewidth}{!}{
    \begin{tabular}{llll|llllllllllll|llllllllllll}
    \cline{3-28}
     & \multicolumn{1}{l|}{} & \multicolumn{2}{c|}{2020} & \multicolumn{12}{c|}{2021} & \multicolumn{12}{c|}{2022} \\ \cline{3-28} 
     & \multicolumn{1}{l|}{} & \multicolumn{1}{c}{J - N} & \multicolumn{1}{c|}{D} & \multicolumn{1}{c}{J} & \multicolumn{1}{c}{F} & \multicolumn{1}{c}{M} & \multicolumn{1}{c}{A} & \multicolumn{1}{c}{M} & \multicolumn{1}{c}{J} & \multicolumn{1}{c}{J} & \multicolumn{1}{c}{A} & \multicolumn{1}{c}{S} & \multicolumn{1}{c}{O} & \multicolumn{1}{c}{N} & \multicolumn{1}{c|}{D} & \multicolumn{1}{c}{J} & \multicolumn{1}{c}{F} & \multicolumn{1}{c}{M} & A & M & J & J & A & S & O & N & D \\ \hline
     & \textbf{Disciplinas - cumprimento de créditos} & \textbf{\textgreater{}\textgreater{}\textgreater{}} & \textbf{\textgreater{}} & \textbf{\textgreater{}} & \textbf{\textgreater{}} & \textbf{\textgreater{}} & \textbf{\textgreater{}} & \textbf{\textgreater{}} &  &  &  &  &  &  &  &  &  &  &  &  &  &  &  &  &  &  &  \\ \hline
     & \textbf{Fundamentação} &  & \textbf{\textgreater{}} & \textbf{\textgreater{}} & \textbf{\textgreater{}} & \textbf{\textgreater{}} & \textbf{\textgreater{}} & \textbf{\textgreater{}} & \textbf{\textgreater{}} &  &  &  &  &  &  &  &  &  &  &  &  &  &  &  &  &  &  \\ \hline
    1 & Fazer revisão bibliográfica &  & \textgreater{} & \textgreater{} &  &  &  &  &  &  &  &  &  &  &  &  &  &  &  &  &  &  &  &  &  &  &  \\
    2 & Elaborar estudo do estado da arte &  &  &  & \textgreater{} & \textgreater{} &  &  &  &  &  &  &  &  &  &  &  &  &  &  &  &  &  &  &  &  &  \\
    3 & Comparar modelos pesquisados &  &  &  &  & \textgreater{} &  &  &  &  &  &  &  &  &  &  &  &  &  &  &  &  &  &  &  &  &  \\
    4 & Estudar odometria visual &  &  &  &  & \textgreater{} & \textgreater{} &  &  &  &  &  &  &  &  &  &  &  &  &  &  &  &  &  &  &  &  \\
    5 & {\color[HTML]{3166FF} \textit{Escrever parte.1 da tese (introdução, fundamentação e metodologia)}} &  &  &  & \textgreater{} & \textgreater{} & \textgreater{} & \textgreater{} & \textgreater{} &  &  &  &  &  &  &  &  &  &  &  &  &  &  &  &  &  &  \\
    6 & {\color[HTML]{9A0000} Realizar qualificação} &  &  &  &  &  &  &  & \textgreater{} &  &  &  &  &  &  &  &  &  &  &  &  &  &  &  &  &  &  \\ \hline
     & \textbf{Desenvolvimento} &  &  &  &  &  &  &  & \textbf{\textgreater{}} & \textbf{\textgreater{}} & \textbf{\textgreater{}} & \textbf{\textgreater{}} & \textbf{\textgreater{}} & \textbf{\textgreater{}} & \textbf{\textgreater{}} &  &  &  &  &  &  &  &  &  &  &  &  \\ \hline
    1 & Realizar simulação dos modelos e odometria &  &  &  &  &  &  &  & \textgreater{} & \textgreater{} & \textgreater{} &  &  &  &  &  &  &  &  &  &  &  &  &  &  &  &  \\
    2 & Elaborar conceito do novo modelo &  &  &  &  &  &  &  &  &  & \textgreater{} & \textgreater{} &  &  &  &  &  &  &  &  &  &  &  &  &  &  &  \\
    3 & Realizar simulação integrada &  &  &  &  &  &  &  &  &  &  & \textgreater{} & \textgreater{} &  &  &  &  &  &  &  &  &  &  &  &  &  &  \\
    4 & {\color[HTML]{3166FF} \textit{Escrever primeiro artigo}} &  &  &  &  &  &  &  &  &  &  &  & \textgreater{} & \textgreater{} & \textgreater{} &  &  &  &  &  &  &  &  &  &  &  &  \\
    5 & {\color[HTML]{3166FF} \textit{Escrever patente}} &  &  &  &  &  &  &  &  &  &  &  & \textgreater{} & \textgreater{} & \textgreater{} &  &  &  &  &  &  &  &  &  &  &  &  \\ \hline
     & \textbf{Resultados e análise} &  &  &  &  &  &  &  &  &  &  &  &  & \textbf{\textgreater{}} & \textbf{\textgreater{}} & \textbf{\textgreater{}} & \textbf{\textgreater{}} & \textbf{\textgreater{}} & \textbf{\textgreater{}} & \textbf{\textgreater{}} & \textbf{\textgreater{}} &  &  &  &  &  &  \\ \hline
    1 & Realizar testes em laboratório &  &  &  &  &  &  &  &  &  &  &  &  & \textgreater{} & \textgreater{} & \textgreater{} & \textgreater{} &  &  &  &  &  &  &  &  &  &  \\
    2 & Analisar os dados obtidos &  &  &  &  &  &  &  &  &  &  &  &  &  &  &  & \textgreater{} & \textgreater{} &  &  &  &  &  &  &  &  &  \\
    3 & {\color[HTML]{3166FF} \textit{Escrever segundo artigo}} &  &  &  &  &  &  &  &  &  &  &  &  &  &  &  &  &  & \textgreater{} & \textgreater{} & \textgreater{} &  &  &  &  &  &  \\ \hline
     & \textbf{Defesa} &  &  &  &  &  &  &  &  &  &  &  &  &  &  &  &  &  &  &  & \textbf{\textgreater{}} & \textbf{\textgreater{}} & \textbf{\textgreater{}} & \textbf{\textgreater{}} & \textbf{\textgreater{}} & \textbf{\textgreater{}} & \textbf{\textgreater{}} \\ \hline
    1 & {\color[HTML]{3166FF} \textit{Escrever parte.2 da tese (Resultados e análise e conclusão)}} &  &  &  &  &  &  &  &  &  &  &  &  &  &  &  &  &  &  &  & \textgreater{} & \textgreater{} & \textgreater{} & \textgreater{} & \textgreater{} &  &  \\
    2 & {\color[HTML]{9A0000} Realizar defesa perante a banca} &  &  &  &  &  &  &  &  &  &  &  &  &  &  &  &  &  &  &  &  &  &  &  &  & \textgreater{} & \textgreater{}
    \end{tabular}
    }
\end{table}

    % \end{columns}

%*----------- notes
    \note[item]{Notes can help you to remember important information. Turn on the notes option.}
\end{frame}
%-
\begin{frame}[c]{Referências}
    %\transboxin[duration=1,direction=30]
    %\centering
    %\bibliographystyle{apalike}
%     \textcite{agostini2007}
%     \parencite{agostini2007}
%     \footcite{agostini2007}
%     \cite{agostini2007}\\
% começo
%     \bibentry{agostini2007}
    \begin{columns}
    \column{.01\textwidth}
    \column{.95\textwidth}
    \Wider{
   
    GUANGYI, Z. et al. Research on underwater safety inspection and operational robot motion control. In: IEEE. 2018 33rd Youth Academic Annual Conference of Chinese Association of Automation (YAC). [S.l.], 2018. p. 322–327.\\
    \vspace*{0.2cm}

    BRUNO, F. et al. Augmented reality visualization of scene depth for aiding rov pilots in underwater manipulation. Ocean Engineering, Elsevier, v. 168, p. 140–154, 2018.\\
    \vspace*{0.2cm}

    LEBORNE, F. et al. Dynamic modeling and identification of an heterogeneously actuated underwater manipulator arm. In: IEEE. 2018 IEEE International Conference on Robotics and Automation (ICRA). [S.l.], 2018. p. 1–9.\\
    \vspace*{0.2cm}

    BARBIERI, L. et al. Design, prototyping and testing of a modular small-sized underwater robotic arm controlled through a master-slave approach. Ocean Engineering, Elsevier, v. 158, p. 253–262, 2018.


    }
    \end{columns}
%*----------- notes
    \note[item]{Notes can help you to remember important information. Turn on the notes option.}
\end{frame}
%-
\begin{frame}[c]{Referências}
    %\transboxin[duration=1,direction=30]
    %\centering
    \begin{columns}
        \column{.01\textwidth}
        \column{.95\textwidth}
        \Wider{
       
        
    
        KURUMAYA, S. et al. A modular soft robotic wrist for underwater manipulation. Soft robotics, Mary Ann Liebert, Inc. 140 Huguenot Street, 3rd Floor New Rochelle, NY 10801 USA, v. 5, n. 4, p. 399–409, 2018.\\
        \vspace*{0.2cm}
    
        SIVČEV, S. et al. Underwater manipulators: A review. Ocean Engineering, Elsevier, v. 163, p. 431–450, 2018. \\
        \vspace*{0.2cm}
    
        SIVČEV, S. et al. Collision detection for underwater rov manipulator systems. Sensors, Multidisciplinary Digital Publishing Institute, v. 18, n. 4, p. 1117, 2018.
        }
        \end{columns}
    
    
    
    
    

%*----------- notes
    %\note[item]{Notes can help you to remember important information. Turn on the notes option.}
\end{frame}
%-